\documentclass[fontsize=12pt]{scrartcl}
\usepackage[portuguese]{babel}
\usepackage[colorlinks,allcolors=blue]{hyperref}
\usepackage[utf8]{inputenc}
\usepackage{amsmath}
\usepackage{graphicx}
\newcommand{\dpar}[1]{\left(#1\right)}
\newcommand{\un}[1]{\mathrm{#1}}

\DeclareMathOperator{\sen}{sen}
\DeclareMathOperator{\pr}{pr}

\title{Física Geral I: Lista de exercícios 3}

\author{Data de entrega: 16 de maio de 2018}

\date{}

\begin{document}
\maketitle
\begin{enumerate}
  \item Uma bola de futebol se encontra a uma distância de $5\,\un m$ do gol, o qual tem $1\,\un m$ de altura. Se uma criança chuta a bola com uma velocidade de $10\,\un m/s$ formando um ângulo $\theta$ com a horizontal. Determine o ângulo máximo 
\end{enumerate}
\end{document}
