\documentclass[fontsize=12pt]{scrartcl}
\usepackage[portuguese]{babel}
\usepackage[colorlinks,allcolors=blue]{hyperref}
\usepackage[utf8]{inputenc}
\usepackage{amsmath}
\usepackage{graphicx}
\newcommand{\dpar}[1]{\left(#1\right)}
\newcommand{\un}[1]{\mathrm{#1}}

\DeclareMathOperator{\sen}{sen}
\DeclareMathOperator{\pr}{pr}

\title{Física Geral I: Lista de exercícios 3}

\author{Data de entrega: 16 de maio de 2018}

\date{}

\begin{document}
\maketitle
\begin{enumerate}
\item Maria está na janela do seu quarto, a uma altura de $5\,\un m$,
  quando ouve seu pai fora de casa pedindo para entrar. Preguiçosa,
  Maria decide lançar a chave da casa para seu pai diretamente pela
  janela. Para ter sucesso na entrega, a chave tem que passar por cima
  de um muro de $3\,\un m$ de altura que está a uma distância
  horizontal de $4\,\un m$ da janela. Se Maria lança a chave de forma
  horizontal, qual é a velocidade mínima que ela tem que proporcionar
  à chave para ter sucesso na entrega?
  % \item Uma bola de futebol se encontra a uma distância de
  %   $5\,\un m$ do gol, o qual tem $1\,\un m$ de altura. Se uma
  %   criança chuta a bola com uma velocidade de $10\,\un m/s$
  %   formando um ângulo $\theta$ com a horizontal. Determine o ângulo
  %   máximo
\item Um motoqueiro insano se encontra no terraço de um prédio e quer
  chegar no terraço de um prédio vizinho, separado por uma distância
  horizontal de $10\,\un m$. Para isso, ele fixa uma rampa que faz um
  ângulo de $30^\circ$ com a horizontal na borda do terraço onde ele
  se encontra. Se o terraço vizinho está a uma altura de $2\,\un m$ em
  relação ao ponto mais alto da rampa, qual é a velocidade mínima com
  a qual a moto deve deixar a rampa para que o motoqueiro não tenha
  uma queda fatal?
\item No exercício anterior, se a moto deixa a rampa com uma
  velocidade de $20\,\un m/\un s$, determine o módulo da velocidade
  com a que chega no terraço vizinho.
\item Uma plataforma inclinada faz um ângulo de $30^\circ$ com a
  horizontal. Do ponto mais baixo da plataforma é jogada uma pedra com
  uma velocidade inicial de $10\,\un m/\un s$ fazendo um ângulo de
  $30^\circ$ em relação à plataforma. Determine a distância entre o
  ponto inicial e o ponto em que a pedra colide com a plataforma.
\item Em um parque de diversões, João e Maria sobem em uma plataforma
  circular e sentam em lugares diferentes. Jõao está a uma distância
  de $3\,\un m$ do centro da plataforma e Maria a $1,5\,\un m$. A
  plataforma gira com velocidade angular constante e ambos notam que
  deram 4 voltas em $1\,\un{min}$. Determine
  \begin{enumerate}
  \item a velocidade angular da plataforma em $\un{rad}/\un s$;
  \item o módulo das velocidades de Jõao e de Maria em $\un m/\un s$;
  \item as acelerações centrípetas de Jõao e de Maria em
    $\un m/\un{s}^2$.
  \end{enumerate}
\item Um disco de raio $0,2\,\un m$ gira em torno do seu centro com
  velocidade angular em $\un{rad}/\un s$ dada pela equação
  $\omega(t)=0.5t^3$ ($t$ em segundos). Determine as acelerações
  tangencial e centrípeta de um ponto na borda do disco no instante
  $t=2\,\un s$.
\item Uma pessoa vai atravessar um rio de $200\,\un m$ de largura
  usando um bote. Suponha que o bote sempre se move com velocidade
  constante de módulo igual a $10\,\un m/\un s$ em relação à
  agua. Além disso, considere que o rio corre com uma velocidade
  constante de $5\,\un m/\un s$ em relação a terra firme. Faça o
  seguinte:
  \begin{enumerate}
  \item Se o bote em todo momento está orientado perpendicularmente à
    correnteza, determine o tempo que o bote demora em atravessar o
    rio.
  \item Determine a distância que o bote percorreu no item anterior.
  \item Encontre a direção que o bote debe manter para que sua
    trajetória em relação a terra firme seja perpendicular à
    correnteza.
  \end{enumerate}
\item João se encontra sobre uma plataforma rente ao chão que se move
  em linha reta com velocidade constante de $2\,\un m/\un s$ em
  relação a Maria, que está fora da plataforma. Quando Jõao passa por
  Maria, ele lança uma moeda verticalmente desde uma altura de
  $1\,\un m$ com uma velocidade de $4\,\un m/\un s$ e volta a pegá-la
  na mesma altura. Determine a expressão da posição da moeda em
  qualquer instante de tempo em relação a Maria. Qual é a trajetória
  da moeda segundo Maria?
\end{enumerate}
\end{document}
