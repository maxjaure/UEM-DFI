\documentclass[12pt, a4paper]{article}
\usepackage[portuguese]{babel}
\usepackage[colorlinks,allcolors=blue]{hyperref}
\usepackage[utf8]{inputenc}

\newcommand{\dpar}[1]{\left(#1\right)}
\newcommand{\un}[1]{\mathrm{#1}}

\title{Vetores}

\author{Max Jáuregui}
\begin{document}
\maketitle

\section{Introdução}

A física é uma ciência fundamental, cujo principal objeto de estudo são os fenômenos naturais, também chamados de fenômenos físicos. O estudo desses fenômenos dá origem a teorias e princípios gerais, os quais são importantes para o próprio desenvolvimento da física assim como para eventuais aplicações práticas. 

A física está sempre em constante desenvolvimento. Por exemplo, no século XVI, após várias observações da queda livre de objetos, Galileu propôs que, desprezando a resistência do ar, \textit{a aceleração de um corpo em queda livre não depende so seu peso}. Isso pode ser verificado atualmente em uma câmera de vácuo (veja \url{https://www.youtube.com/watch?v=hRkbx0YbHfU}). No entanto, esse princípio somente é valido para corpos que caem de alturas pequenas quando comparadas com o raio da Terra. De fato, poucos anos depois da morte de Galileu, Newton desenvolveu a teoria da gravitação universal, a qual inclui a afirmação de Galileu como um caso particular aproximado. Pela sua vez, a teoria de Newton tem também um limite de validade e é um caso particular aproximado da teoria da relatividade geral de Einstein, proposta quase 200 anos depois da morte de Newton. 

A física estuda os fenômenos físicos de forma qualitativa e quantitativa fazendo uso da linguagem da matemática. Para isso, na observação desses fenômenos são obtidos dados numéricos por meio de medições ou cálculos. Chamamos de uma \textbf{grandeza física} a uma quantidade numérica associada a um fenômeno físico; por exemplo, a massa de um corpo, o lapso de tempo entre dois eventos, a velocidade de um corpo, etc..

Para medir uma grandeza física associada a um sistema físico precisamos compará-la com um padrão, o qual define uma \textbf{unidade} da grandeza.

No nosso curso, usaremos o sistema internacional de unidades (SI):
\begin{itemize}
	\item Unidade de tempo: segundo (s).
	\item Unidade de comprimento: metro (m).
	\item Unidade de massa: quilograma (kg). 
\end{itemize}
O grama (g) não é unidade fundamental de massa segundo o SI.

Para denotar múltiplos e frações de unidades, usamos prefixos:

\begin{center}
	\begin{tabular}{c|c|c}
		Prefixo&Símbolo&Fator\\\hline
		quilo&k&$10^3$\\
		centi&c&$10^{-2}$\\
		mili&m&$10^{-3}$\\
		micro&$\mu$&$10^{-6}$\\
		nano&n&$10^{-9}$
	\end{tabular}
\end{center}
Por exemplo, $1\un{cm}=10^{-2}\un{m}$, $5\mu\un{g}=5\times10^{-6}\un{g}$.

Exemplos de conversão de unidades: 
\begin{itemize}
	\item Se o velocímetro de um carro marca 90 km/h, qual é a velocidade do carro em m/s? Vemos que
	$$90 \frac{\un{km}}{\un{h}}=\dpar{90 \frac{\un{km}}{\un{h}}}\dpar{\frac{1000\un{m}}{1\un{km}}}\dpar{\frac{1\un{h}}{3600\un{s}}}=25\frac{\un{m}}{\un{s}}\,.$$
	\item Um mililitro (mL) equivale a $1\un{cm}^3$, a quantos litros equivale $1\un{m}^3$? Vemos que
	$$1\un{m}^3=1\un{m}^3\dpar{\frac{100\un{cm}}{1\un{m}}}^3\dpar{\frac{1\un{mL}}{1\un{cm}^3}}\dpar{\frac{1\un{L}}{1000\un{mL}}}=1000\un{L}\,.$$
\end{itemize}


\section{Adição de vetores}
Uma grandeza física é dita uma \textbf{grandeza escalar} quando é descrita por um único número; por exemplo, o lapso de tempo entre dois eventos é um exemplo de uma grandeza escalar. Outra classe importante de grandezas físicas, chamadas de \textbf{grandezas vetoriais}, não podem ser descritas por um único número. Por exemplo, para descrever a velocidade de um corpo, precisamos saber quão rápido ele se move (módulo da velocidade) e em qual direção. Grandezas vetoriais podem ser denotadas graficamente por segmentos de reta orientados (setas), os quais chamaremos de \textbf{vetores}.

\end{document}