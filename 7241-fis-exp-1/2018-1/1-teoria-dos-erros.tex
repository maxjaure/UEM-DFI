\documentclass[12pt, a4paper]{article}
\usepackage[portuguese]{babel}
\usepackage[colorlinks,allcolors=blue]{hyperref}
\usepackage[utf8]{inputenc}
\usepackage{amsmath}

\newcommand{\dpar}[1]{\left(#1\right)}
\newcommand{\un}[1]{\mathrm{#1}}

\title{Introdução à teoria dos erros}

\author{Max Jáuregui}
\begin{document}
\maketitle

Uma \textbf{grandeza física} é qualquer quantidade numérica associada a um fenômeno físico; por exemplo, a distância entre dois corpos, o lapso de tempo entre dois eventos, a velocidade de um corpo, etc. Para obtermos um valor associado a uma grandeza física, precisamos realizar um \textbf{processo de medição}. A medição de uma grandeza física pode ser direta ou indireta.

\paragraph{Medição direta:} A medição consiste em comparar a grandeza física diretamente com um padrão utilizando um instrumento de medição. Podemos realizar uma medição direta de duas formas:
\begin{itemize}
	\item Fazendo uma única medição: Por exemplo, podemos medir o comprimento de uma mesa com uma régua cuja menor subdivisão tem comprimento igual a 1mm.
	\item Considerando várias repetições da medição: Por exemplo, podemos medir repetidamente com um cronômetro o tempo que demora em cair um corpo desde uma certa altura até o chão. É claro que as repetições do experimento devem estar sujeitas às mesmas condições.
\end{itemize}

\paragraph{Medição indireta:}
A medida é obtida utilizando uma equação que envolve grandezas físicas que podem ser medidas diretamente. Por exemplo, para medir o módulo da velocidade média de um corpo que cai livremente de uma altura $h$ em relação ao chão, usamos a relação $v=h/t$, onde $t$ é o tempo que demora o corpo em cair até o chão.

Mesmo que sejamos experientes no processo de medição de uma grandeza física, as medidas obtidas sempre apresentarão erros. Podemos classificar os erros em: erros sistemáticos, erros aleatórios.

\paragraph{Erros aleatórios:} São erros causados principalmente pela ignorância inevitável das condições exatas de medição. Esses erros são imprevisíveis e não podem ser diminuídos.

\paragraph{Erros sistemáticos:} São erros causados pelo uso de instrumentos não calibrados ou pelo mau costume da pessoa que faz a medição. Esses erros podem ser diminuídos.

Para explicitar o erro em uma medição de uma grandeza física, a medida deve ser expressa na forma $\overline{x}\pm \sigma_x$. Essa notação quer dizer que, com certeza ou pelo menos com grande probabilidade ($\ge 68\%$), o valor da grandeza se encontra no intervalo $[\overline{x}-\sigma_x, \overline{x}+\sigma_x]$. No caso de uma medida direta fazendo uma única medição, $\overline{x}$ e $\sigma_x$ são respectivamente a leitura e a \textbf{incerteza} do instrumento de medição. A incerteza do instrumento é as vezes fornecida pelo fabricante; caso contrário, podemos considerar que a incerteza do instrumento é igual à menor subdivisão do instrumento. No caso de uma medida direta considerando $n$ repetições da medição, vamos ter 
$$\overline{x}=\frac{x_1+\cdots+x_n}{n}\quad\text{(média aritmética)}\,,$$
onde $x_1,\ldots,x_n$ são as leituras (sem incerteza) do instrumento de medição nas repetições. Por outro lado,
$$\sigma_x=\sqrt{\frac{(x_1-\overline{x})^2+\cdots+(x_n-\overline{x})^2}{n-1}}\,,$$
que é chamado de \textbf{desvio padrão} de $x_1,\ldots,x_n$.

Sejam $\overline{x}+\sigma_x$ e $\overline{y}+\sigma_y$ as medidas de duas grandezas. Se $z=x+y$, como determinamos $\sigma_z$? O maior valor de $z$ será $x+y+\sigma_x+\sigma_y$ e o menor valor será $x+y-\sigma_x-\sigma_y$. Logo, é seguro considerar $\sigma_z=\sigma_x+\sigma_y$. Porém isso superestima o erro da medida de $z$. Se $x$ e $y$ são medidas independentes, podemos considerar $\sigma_z=\sqrt{\sigma_x^2+\sigma_y^2}$.

Sejam $\overline{x}+\sigma_x$ e $\overline{y}+\sigma_y$ as medidas de duas grandezas. Se $z=xy$, como determinamos $\sigma_z$? Primeiramente consideramos que $\sigma_x\ll x$ e $\sigma_y\ll y$. Logo, o maior valor de $z$ será $z+\sigma_z=(x+\sigma_x)(y+\sigma_y)$.

\end{document}