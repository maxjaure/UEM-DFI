\documentclass[12pt,a4paper]{article}
\usepackage{microtype}
\usepackage{amsfonts}
\usepackage{amsthm}
\usepackage{graphicx}

\newcommand{\N}{\mathbb{N}}
\newcommand{\R}{\mathbb{R}}

\theoremstyle{definition}
\newtheorem{ex}{Exercise}[section]

\title{Notes of elementary physics 1}
\author{Max Jauregui}

\begin{document}
\maketitle
\section{Kinematics in one dimension}

\subsection{Frames of reference}

We are going to study the motion of an object which can move along a
straight line. We assume that the dimensions of the object are not
important in its motion. Hence, any object will be considered as a
\emph{particle}, which is represented by a dot.

In order to study the motion of a particle, we need to define a
\emph{frame of reference}, which is composed by a coordinate system
and a clock. In the particular case where a particle moves along a
straight line, the coordinate system is just composed by a point,
labeled as the \emph{origin} of the frame of reference, and an axis,
usually called the \emph{$x$-axis}.

Fixed a frame of reference, the position of a particle at an instant
$t\ge 0$ is given by a real number $x(t)$. This defines a function
$x:[0,\infty)\to\R$, which, as almost any function in physics, will be
assumed to have derivatives of any order.

The international system of units (SI) establishes the second (s) as
the unit of time and the meter (m) as the unit of length. In this
direction, we will consider that $x(t)$ is the position of a particle
measured in meters at an instant $t$, measured in seconds.

\subsection{Average velocity and average speed}

Let us consider a particle that moves in a straight line and let us
suppose that we know the position of the particle at two distinct
instants $t_1$ and $t_2$. The \emph{average velocity} of the particle
in this interval of time is defined as
$$\overline v=\frac{x(t_2)-x(t_1)}{t_2-t_1}\,.$$
Introducing the notation $\Delta t=t_2-t_1$, we can write
$$\overline{v}=\frac{x(t_1+\Delta t)-x(t_1)}{\Delta t}\,.$$
The unit of average velocity in the SI is meters per second (m/s).

The average velocity of a particle can be positive, negative or
zero. For instance, if $x(t_2)=x(t_1)$ (the particle returns to the
same position), then $\overline{v}=0\,\mathrm{m/s}$. In this case, the
particle is not necessarily at rest and, consequently, it could have
traveled a distance $d$ different from zero. We define the
\emph{average speed} of the particle as
$$v_s=\frac{d}{|\Delta t|}\,.$$
The average speed of a particle is always non-negative. The definition
of average speed remains the same for the general motion of a particle
in three dimensions.
\begin{ex}
  A dog is initially (instant $0\,\mathrm{s}$) in the position
  $0\,\mathrm{m}$. Then, the dog runs $40\,\mathrm{m}$ to the right
  and then $20\,\mathrm{m}$ to the left, reaching its final position
  at the instant $12\,\mathrm{s}$. Find the average velocity and the
  average speed of the dog in this interval of time. \emph{Answer:}
  $\overline{v}=1{,}7\,\mathrm{m/s}$ and
  $\overline{v}_s=5\,\mathrm{m/s}$.
\end{ex}

\subsection{Instantaneous velocity}

Let us begin with an example.

\begin{ex}
  A particle moves in a straight line such that it is in the position
  $1{,}0\,\mathrm{m}$ at instant $1{,}0\,\mathrm{s}$. (i) If the
  particle is in the position $4{,}0\,\mathrm{m}$ at the instant
  $2{,}0\,\mathrm{s}$, find the average velocity of the particle
  between the instants $1{,}0\,\mathrm{s}$ and
  $1{,}2\,\mathrm{s}$. (ii) Suppose that the particle is in the
  positions $2{,}3\,\mathrm{m}$ and $1{,}7\,\mathrm{m}$ at the
  instants $1{,}5\,\mathrm{s}$ and $1{,}1\,\mathrm{s}$
  respectively. Find the average velocity of the particle between the
  instants $1{,}0\,\mathrm{s}$ and $1{,}5\,\mathrm{s}$ and between
  $1{,}0\,\mathrm{s}$ and $1{,}3\,\mathrm{s}$. \emph{Answer:} (i)
  $3{,}0\,\mathrm{m/s}$ (ii) $2{,}6\,\mathrm{m/s}$ and
  $2{,}3\,\mathrm{m/s}$.
\end{ex}

The exercise above illustrates that the average velocity between the
instants $1\,\mathrm{s}$ and $t$ approximate to $2\,\mathrm{m/s}$ as
$t$ approaches $1\,\mathrm{s}$. This motivates to say that the
velocity of the particle at the instant $1\,\mathrm{s}$ is
$2\,\mathrm{m/s}$. This velocity is called the \emph{instantaneous
  velocity} and is defined in general, for an arbitrary instant $t$,
by
$$v(t)=\lim_{\Delta t\to 0}\frac{x(t+\Delta t)-x(t)}{\Delta t}\,.$$
The limit on the right hand side of this equation is a very important
one and is called the \emph{derivative} of the function $x$ at the
point $t$, denoted by $dx/dt$. Then, we have
$$v(t)=\frac{dx}{dt}\,.$$
The absolute value of the instantaneous velocity is called
\emph{instantaneous speed}.

In this course we will essentially work with polynomial functions. In
order to compute the derivative of a polynomial, we will use the
following rules:
\begin{eqnarray*}
  \frac{dt^n}{dt}&=&nt^{n-1}\,,\quad n\in\N\,;\\
  \frac{dc}{dt}&=& 0\,,\quad\mbox{for any constant }c\,.
\end{eqnarray*}

\begin{ex}
  The position of a particle in meters is given by the expression
  $x(t)=5 t^2-2 t+3$ if $t$ is measured in seconds. Find the
  instantaneous velocity of the particle at the instant
  $2\,\mathrm{s}$. \emph{Answer:} $18\,\mathrm{m/s}$.
\end{ex}

Having the pairs $(t,x(t))$ for several values of $t$, we can plot the
graph of the function $x$. This graph is usually called
\emph{displacement-time graph} or simply \emph{$x$ vs $t$ graph}. The
ratio $r=[x(t+\Delta t)-x(t)]/\Delta t$ is the slope of the segment
that connects the points $(t,x(t))$ and $(t+\Delta t,x(t+\Delta
t))$. If $\Delta t$ tends to $0$ ($\Delta t\to 0$), the ratio $r$
becomes the slope of the tangent line that passes through the point
$(t,x(t))$. Thus, if we have an $x$ vs $t$ graph, the instantaneous
velocity at an instant $t$ will be given by the slope of the line that
is tangent to the curve at that instant.

\begin{ex}
  Consider the $x$ vs $t$ graph given in Fig.~\ref{fig:xtgraph}. (i)
  Find the approximate values of the instants where the instant
  velocity is $0\,\mathrm{m/s}$. (ii) Find the approximate instant
  where the particle moves from right to left at maximum instantaneous
  speed.

  \begin{figure}[ht]
    \centering
    \includegraphics[width=0.5\textwidth,keepaspectratio]{xtgraph.pdf}
    \caption{$x$ vs $t$ graph.}
    \label{fig:xtgraph}
  \end{figure}
\end{ex}

\end{document}