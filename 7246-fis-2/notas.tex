\documentclass[twocolumn=on,DIV=calc]{scrartcl}
\usepackage[portuguese]{babel}
\usepackage[utf8]{inputenc}
\usepackage[colorlinks,allcolors=blue]{hyperref}
\usepackage{microtype}
\usepackage{amsmath,amsfonts,amsthm}

\newcommand{\dpar}[1]{\left(#1\right)}
\newcommand{\dsqr}[1]{\left[#1\right]}

\theoremstyle{definition}
\newtheorem{ex}{Exemplo}[section]

\DeclareMathOperator{\sen}{sen}

\title{Notas de Física Geral 2}

\author{Max Jáuregui}

\begin{document}
\maketitle
\tableofcontents

\section{Equilíbrio de um corpo rígido}

\subsection{Torque e momento angular}

Considere uma partícula de massa $m$ na posição $\vec r$ sobre a qual
atua uma força $\vec F$. O \textit{torque} da força $\vec F$ é
definido por
\begin{equation}
  \label{eq:1}
  \vec\tau=\vec r\times\vec F\,,
\end{equation}
onde $\times$ denota o produto vetorial de dois vetores, o qual é
definido por
$$\vec A\times\vec B=
\left|
  \begin{array}{ccc}
    \hat i&\hat j&\hat k\\
    A_x&A_y&A_z\\
    B_x&B_y&B_z
  \end{array}
\right|\,.$$ A unidade do torque no sistema internacional (SI) é
$\mathrm{N}\cdot\mathrm{m}$.

\begin{ex}
  Considere uma partícula de $10\,\mathrm{kg}$ na posição
  $\vec r=\cos 30^\circ\hat i+\sen 30^\circ\hat j$. O torque do peso
  da partícula é dado por $\vec\tau=\vec r\times\vec P$, onde
  $\vec P=-98\hat j$. Logo,
  $$\vec\tau=\left|
  \begin{array}{ccc}
    \hat i&\hat j&\hat k\\
    \sqrt{3}/2&1/2&0\\
    0&-98&0
  \end{array}
\right|=-49\sqrt{3}\hat k\,.$$  
\end{ex}

Pela segunda lei de Newton, sabemos que $\vec F=\dot{\vec p}$, onde
$\vec p=m\vec v$ é o momento linear da partícula ($\vec v$ é a
velocidade da partícula). Logo, a Eq.~(\ref{eq:1}) assume a forma
$$\vec\tau=\vec r\times\dot{\vec p}\,.$$
Usando a identidade
\begin{equation}
  \label{eq:2}
  \frac{d}{dt}(\vec r\times\vec p)=\dot{\vec r}\times\vec p+\vec r\times\dot{\vec p}
\end{equation}
e o fato de que $\dot{\vec r}\times\vec p=\vec 0$ (devido a que
$\dot{\vec r}$ e $\vec p$ são paralelos), obtemos que
\begin{equation}
  \label{eq:3}
  \vec\tau=\frac{d}{dt}(\vec r\times\vec p)\,.
\end{equation}

Definimos o \textit{momento angular} da partícula por
$$\vec L=\vec r\times\vec p\,.$$
Usando esse conceito, a Eq.~(\ref{eq:3}) nos diz então que
\begin{equation}
  \label{eq:4}
  \vec\tau=\dot{\vec L}\,,
\end{equation}
ou seja, o torque da força $\vec F$ é igual à taxa de variação do
momento angular da partícula (note a semelhança com a segunda lei de
Newton). A unidade de momento angular no SI é
$\mathrm{kg}\cdot\mathrm{m^2/s}$.

\subsection{Sistemas de partículas}
Considere duas partículas de massas $m_1$ e $m_2$ sobre as quais atuam
forças externas $\vec F_1$ e $\vec F_2$ respectivamente. Além
das forças externas, há forças internas $\vec F_{12}$ (sobre a
partícula $1$) e $\vec F_{21}$ (sobre a partícula $2$) devido a
interação das partículas (por exemplo, interação gravitacional ou
elétrica), as quais são paralelas ao vetor
$\vec r_1-\vec r_2$. Pela segunda lei de Newton vamos ter
então que
\begin{equation}
  \label{eq:5}
  \begin{split}
    \vec F_1+\vec F_{12}&=\dot{\vec p}_1\\
    \vec F_2+\vec F_{21}&=\dot{\vec p}_2\,.
  \end{split}
\end{equation}
Somando essas equações, temos que
$$\vec F_1+\vec F_2+\vec F_{12}+\vec F_{21}=\dot{\vec p}_1+\dot{\vec p}_2\,.$$
Como $\vec F_{12}=-\vec F_{21}$ pela terceira lei de Newton, a
equação anterior pode ser escrita como
\begin{equation}
  \label{eq:6}
  \vec F_{\mathrm{ext}}=\dot{\vec p}\,,
\end{equation}
onde $\vec F_{\mathrm{ext}}=\vec F_1+\vec F_2$ é a força
externa total sobre o sistema composto pelas partículas $1$ e $2$, e
$\vec p=\vec p_1+\vec p_2$ é o momento linear total do
sistema.

Multiplicando vetorialmente as Eqs.~(\ref{eq:5}) pelos vetores posição
$\vec r_1$ e $\vec r_2$ respectivamente, vamos ter que
\begin{equation*}
  \begin{split}
    \vec r_1\times(\vec F_1+\vec F_{12})&=\vec r_1\times\dot{\vec p}_1\\
    \vec r_2\times(\vec F_2+\vec F_{21})&=\vec r_2\times\dot{\vec p}_2\\
  \end{split}
\end{equation*}
Somando essas duas equações, vamos ter que
$$\vec r_1\times\vec F_1+\vec r_2\times\vec F_2+\vec r_1\times\vec F_{12}+\vec r_2\times\vec F_{21}=\vec r_1\times\dot{\vec p}_1+\vec r_2\times\dot{\vec p}_2\,.$$
Usando a terceira lei de Newton e a identidade~(\ref{eq:2}) vamos ter
que
$$\vec r_1\times\vec F_1+\vec r_2\times\vec F_2+(\vec r_1-\vec r_2)\times\vec F_{12}=\frac{d}{dt}(\vec r_1\times\vec p_1+\vec r_2\times\vec p_2)\,.$$
Finalmente, como $\vec F_{12}$ é paralelo a
$\vec r_1-\vec r_2$,
$(\vec r_1-\vec r_2)\times\vec F_{12}=\vec 0$ e, por
conseguinte,
\begin{equation}
  \label{eq:7}
  \vec\tau_{\mathrm{ext}}=\dot{\vec L}\,,
\end{equation}
onde
$\vec\tau_{\mathrm{ext}}=\vec\tau_1+\vec\tau_2$
é o torque externo total sobre o sistema e
$\vec L=\vec L_1+\vec L_2$ é o momento angular total do
sistema.

No caso de sistemas compostos por mais de duas partículas, as Eqs.~(\ref{eq:6}) e~(\ref{eq:7}) continuam valendo.

\subsection{Dinâmica do corpo rígido}

Um \textit{corpo rígido} é um corpo não pontual que não se deforma. O
fato de um corpo rígido não se deformar equivale a dizer que a
distância entre dois pontos quaisquer do corpo é uma constante.

Para estudar a dinâmica de um corpo rígido, podemos dividir ele em $n$
partes pequenas e assim considerar o corpo rígido como um sistema de
$n$ corpos. Se a massa da $i$-ésima parte é $\Delta m_i$, sua posição
é $\vec r_i$ e sua velocidade é $\vec v_i$, o momentum total do
sistema é
$$\vec p=\sum_{i=1}^n\vec v_i\Delta m_i$$
e o momento angular total do sistema é
$$\vec L=\sum_{i=1}^n(\vec r_i\times\vec v_i)\Delta m_i\,.$$
Considerando que o número de partes $n$ tende ao infinito e
simultaneamente a massa de cada parte tende a zero, as somas
anteriores tornam-se integrais. Dessa maneira, encontramos que o
\textit{momento linear do corpo rígido} é dado por
\begin{equation}
  \label{eq:8}
  \vec p=\int \vec v\,dm
\end{equation}
e o \textit{momento angular do corpo rígido} por
\begin{equation}
  \label{eq:9}
  \vec L=\int (\vec r\times\vec v)\,dm\,,
\end{equation}
onde as integrais são sobre toda a massa do corpo rígido.

As Eqs.~(\ref{eq:6}) e~(\ref{eq:7}) continuam valendo da mesma
forma para um corpo rígido, levando em conta que $\vec p$ e $\vec L$
são dadas pelas Eqs.~(\ref{eq:8}) e~(\ref{eq:9}).

O movimento de um corpo rígido pode ser estudado analisando
separadamente os movimentos de translação e de rotação em torno de um
eixo que passa pelo corpo.

\subsubsection{Movimento de translação}
Quando um corpo rígido realiza um movimento de translação pura, todos
os pontos do corpo possuem a mesma velocidade. Dessa forma, a
velocidade $\vec v$ pode sair da integral na Eq.~(\ref{eq:8}) e assim o momento
linear do corpo rígido vai ser dado por
\begin{equation}
  \label{eq:10}
  \vec p=\vec v\int dm=M\vec v\,,
\end{equation}
onde $M$ é a massa total do corpo rígido. De forma análoga, na
Eq.~(\ref{eq:9}) vamos ter
\begin{equation}
  \label{eq:11}
  \vec L=\dpar{\int \vec r\,dm}\times\vec v\,.
\end{equation}
Definindo a posição do \textit{centro de massa} do corpo rígido por
$$\vec R=\frac{1}{M}\int \vec r\,dm\,,$$
a Eq.~(\ref{eq:11}) pode ser escrita como
\begin{equation}
  \label{eq:12}
  \vec L=M\vec R\times\vec v=\vec R\times\vec p\,.
\end{equation}

As Eqs.~(\ref{eq:10}) e~(\ref{eq:12}) nos dizem que um corpo rígido
que realiza um movimento de translação pura se comporta como uma
partícula de massa $M$ localizada no centro de massa do corpo. em
outras palavras, as dimensões do corpo rígido não são relevantes no
movimento de translação.

\subsubsection{Movimento de rotação}
Diferentemente do caso do movimento de translação, quando um corpo
rígido realiza um movimento de rotação ao redor de um eixo que passa
por ele, os pontos do corpo que estão mais próximos do eixo de rotação
tem velocidade menor do que os pontos mais afastados. Logo, nesse caso
a velocidade $\vec v$ não pode sair da integral nas Eqs.~(\ref{eq:8})
e~(\ref{eq:9}).

Para simplificar nosso estudo do movimento de rotação de um corpo
rígido, vamos considerar um corpo homogêneo (massa distribuída
uniformemente) e simétrico. Além disso, vamos analisar o caso em que o
corpo gira em torno de um dos seus eixos de simetria com velocidade
angular $\vec\omega$.

Antes de continuar, lembramos que, para uma partícula que realiza um
movimento circular com velocidade angular $\vec\omega$, vale a relação
vetorial $\vec v=\vec\omega\times\vec r$, onde $\vec v$ é a velocidade
da partícula e $\vec r$ é sua posição em relação a um sistema de
referência fixo ao eixo de rotação.

Continuando nossa análise da rotação de um corpo rígido, vemos que o
momento angular do corpo é dado por
$$\vec L=\int [\vec r\times(\vec\omega\times\vec r)]\,dm\,.$$
Usando a identidade vetorial
$$\vec a\times(\vec b\times\vec c)=(\vec a\cdot\vec c)\vec b-(\vec a\cdot\vec b)\vec c\,,$$
temos que
\begin{equation}
  \label{eq:13}
  \vec L=\int [(\vec r\cdot\vec r)\vec\omega-(\vec r\cdot\vec\omega)\vec r]\,dm\,.
\end{equation}
Se $\theta$ é o ângulo entre os vetores $\vec\omega$ e $\vec r$, então
$\vec r\cdot\omega=r\omega\cos\theta$ e, por conseguinte,
$$\vec L=\int [r^2\vec\omega-(r\omega\cos\theta)\vec r]\,dm\,.$$
Devido a que o corpo gira em torno de um eixo de simetria, podemos
inferir que o vetor $\vec L$ deve ter a mesma direção que o vetor
$\vec\omega$. Isso quer dizer que as direções perpendiculares a
$\vec\omega$ devem se anular após fazer a integral na
Eq.~(\ref{eq:13}). Logo, vamos ter
\begin{equation*}
  \begin{split}
    \vec L&=\int \dpar{r^2\vec\omega-(r\omega\cos\theta)(r\cos\theta)\frac{\vec\omega}{\omega}}\,dm\\
    &=\int (r^2-r^2\cos^2\theta)\vec\omega\,dm\\
    &=\int (r^2\sen^2\theta)\vec\omega\,dm\,.
  \end{split}
\end{equation*}
Como todo ponto do corpo gira com a mesma velocidade angular, obtemos
que
\begin{equation}
  \label{eq:14}
  \vec L=I\vec\omega\,,
\end{equation}
onde
\begin{equation}
  \label{eq:15}
  I=\int (r\sen\theta)^2\,dm
\end{equation}
é o chamado \textit{momento de inércia} do corpo. Cabe ressaltar que
para cada ponto do corpo, $r\sen\theta$ é a distância do ponto ao eixo
de rotação.

A Eq.~(\ref{eq:14}) nos diz que o momento angular é um indicador da
rotação de um corpo rígido.

\subsection{Estática de um corpo rígido}
Um corpo rígido está em \textit{equilíbrio estático} quando não
realiza movimento de translação nem de rotação.

Se um corpo rígido está em equilíbrio estático, todo ponto do corpo
tem velocidade nula. Logo, segue das Eqs.~(\ref{eq:8}) e~(\ref{eq:9})
que nesse caso vamos ter
$$\vec p=\vec 0\quad\text{e}\quad\vec L=\vec 0\,,$$
onde, na segunda condição, $\vec L$ pode ser calculado em relação a
qualquer ponto. Substituindo isso nas Eqs.~(\ref{eq:6})
e~(\ref{eq:7}), obtemos as chamadas condições de equilíbrio para o
corpo rígido:
\begin{equation}
  \label{eq:16}
  {\vec F}_{\mathrm{ext}}=\vec 0\quad\text{e}\quad \vec\tau_{\mathrm{ext}}=\vec 0\,,
\end{equation}
onde, na segunda condição, $\vec\tau_{\mathrm{ext}}$ pode ser
calculado em relação a qualquer ponto.

O fato de um corpo rígido satisfazer uma das condições dadas na
Eq.~(\ref{eq:16}) não implica que a outra será também satisfeita. Por
exemplo, se temos uma barra sobre uma mesa e aplicamos forças de
direções opostas sobre os extremos da barra e perpendiculares a ela, a
primeira condição em~(\ref{eq:16}) é satisfeita, mas a segunda não.
\end{document}