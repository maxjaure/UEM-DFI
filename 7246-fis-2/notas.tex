\documentclass[twocolumn=on,DIV=calc]{scrartcl}
\usepackage[portuguese]{babel}
\usepackage[utf8]{inputenc}
\usepackage[colorlinks,allcolors=blue]{hyperref}
\usepackage{microtype}
\usepackage{amsmath,amsfonts}

\title{Notas de Física Geral 2}

\author{Max Jáuregui}

\begin{document}
\maketitle
\tableofcontents

\section{Equilíbrio de um corpo rígido}

\subsection{Torque e momento angular}

Considere uma partícula de massa $m$ na posição $\mathbf{r}$ sobre a qual atua uma
força $\mathbf{F}$. O \textit{torque} da força $\mathbf{F}$ é definido
por
\begin{equation}
  \label{eq:1}
  \boldsymbol{\tau}=\mathbf{r}\times\mathbf{F}\,,
\end{equation}
onde $\times$ denota o produto vetorial de dois vetores, o qual é
definido por
$$\mathbf{A}\times\mathbf{B}=
\left|
  \begin{array}{ccc}
    \mathbf{e}_x&\mathbf{e}_y&\mathbf{e}_z\\
    A_x&A_y&A_z\\
    B_x&B_y&B_z
  \end{array}
\right|\,.$$
A unidade do torque no sistema internacional (SI) é $\mathrm{N}\cdot\mathrm{m}$.

Pela segunda lei de Newton, sabemos que $\mathbf{F}=\dot{\mathbf{p}}$,
onde $\mathbf{p}=m\mathbf{v}$ é o momento linear da partícula. Logo, a
Eq.~(\ref{eq:1}) assume a forma
$$\boldsymbol{\tau}=\mathbf{r}\times\dot{\mathbf{p}}\,.$$
Usando a identidade
\begin{equation}
  \label{eq:2}
  \frac{d}{dt}(\mathbf{r}\times\mathbf{p})=\dot{\mathbf{r}}\times\mathbf{p}+\mathbf{r}\times\dot{\mathbf{p}}
\end{equation}
e o fato de que $\dot{\mathbf{r}}\times\mathbf{p}=\mathbf{0}$ (devido
a que $\dot{\mathbf{r}}$ e $\mathbf{p}$ são paralelos), obtemos que
\begin{equation}
  \label{eq:3}
  \boldsymbol{\tau}=\frac{d}{dt}(\mathbf{r}\times\mathbf{p})\,.
\end{equation}

Definimos o \textit{momento angular} da partícula por
$$\mathbf{L}=\mathbf{r}\times\mathbf{p}\,.$$
Usando esse conceito, a Eq.~(\ref{eq:3}) nos diz então que
\begin{equation}
  \label{eq:4}
  \boldsymbol{\tau}=\dot{\mathbf{L}}\,,
\end{equation}
ou seja, o torque da força $\mathbf{F}$ é igual à taxa de variação do
momento angular da partícula (note a semelhança com a segunda lei de
Newton). A unidade de momento angular no SI é $\mathrm{kg}\cdot\mathrm{m^2/s}$.

\subsection{Sistemas de partículas}
Considere duas partículas de massas $m_1$ e $m_2$ sobre as quais atuam
forças externas $\mathbf{F}_1$ e $\mathbf{F}_2$ respectivamente. Além
das forças externas, há forças internas $\mathbf{F}_{12}$ (sobre a
partícula $1$) e $\mathbf{F}_{21}$ (sobre a partícula $2$) devido a
interação das partículas (por exemplo, interação gravitacional ou
elétrica), as quais são paralelas ao vetor
$\mathbf{r}_1-\mathbf{r}_2$. Pela segunda lei de Newton vamos ter
então que
\begin{equation}
  \label{eq:5}
  \begin{split}
    \mathbf{F}_1+\mathbf{F}_{12}&=\dot{\mathbf{p}_1}\\
    \mathbf{F}_2+\mathbf{F}_{21}&=\dot{\mathbf{p}_2}\,.
  \end{split}
\end{equation}
Somando essas equações, temos que
$$\mathbf{F}_1+\mathbf{F}_2+\mathbf{F}_{12}+\mathbf{F}_{21}=\dot{\mathbf{p}_1}+\dot{\mathbf{p}_2}\,.$$
Como $\mathbf{F}_{12}=-\mathbf{F}_{21}$ pela terceira lei de Newton, a
equação anterior pode ser escrita como
\begin{equation}
  \label{eq:6}
  \mathbf{F}_{\mathrm{ext}}=\dot{\mathbf{p}}\,,
\end{equation}
onde $\mathbf{F}_{\mathrm{ext}}=\mathbf{F}_1+\mathbf{F}_2$ é a força
externa total sobre o sistema composto pelas partículas $1$ e $2$, e
$\mathbf{p}=\mathbf{p}_1+\mathbf{p}_2$ é o momento linear total do
sistema.

Multiplicando vetorialmente as Eqs.~(\ref{eq:5}) pelos vetores posição
$\mathbf{r}_1$ e $\mathbf{r}_2$ respectivamente, vamos ter que
\begin{equation*}
  \begin{split}
    \mathbf{r}_1\times(\mathbf{F}_1+\mathbf{F}_{12})&=\mathbf{r}_1\times\dot{\mathbf{p}_1}\\
    \mathbf{r}_2\times(\mathbf{F}_2+\mathbf{F}_{21})&=\mathbf{r}_2\times\dot{\mathbf{p}_2}\\
  \end{split}
\end{equation*}
Somando essas duas equações, vamos ter que
$$\mathbf{r}_1\times\mathbf{F}_1+\mathbf{r}_2\times\mathbf{F}_2+\mathbf{r}_1\times\mathbf{F}_{12}+\mathbf{r}_2\times\mathbf{F}_{21}=\mathbf{r}_1\times\dot{\mathbf{p}_1}+\mathbf{r}_2\times\dot{\mathbf{p}_2}\,.$$
Usando a terceira lei de Newton e a identidade~(\ref{eq:2}) vamos ter
que
$$\mathbf{r}_1\times\mathbf{F}_1+\mathbf{r}_2\times\mathbf{F}_2+(\mathbf{r}_1-\mathbf{r}_2)\times\mathbf{F}_{12}=\frac{d}{dt}(\mathbf{r}_1\times\mathbf{p}_1+\mathbf{r}_2\times\mathbf{p}_2)\,.$$
Finalmente, como $\mathbf{F}_{12}$ é paralelo a
$\mathbf{r}_1-\mathbf{r}_2$,
$(\mathbf{r}_1-\mathbf{r}_2)\times\mathbf{F}_{12}=\mathbf{0}$ e, por
conseguinte,
\begin{equation}
  \label{eq:7}
  \boldsymbol{\tau}_{\mathrm{ext}}=\dot{\mathbf{L}}\,,
\end{equation}
onde
$\boldsymbol{\tau}_{\mathrm{ext}}=\boldsymbol{\tau}_1+\boldsymbol{\tau}_2$
é o torque externo total sobre o sistema e
$\mathbf{L}=\mathbf{L}_1+\mathbf{L}_2$ é o momento angular total do
sistema.

No caso de sistemas compostos por mais de duas partículas, as
equações~(\ref{eq:6}) e~(\ref{eq:7}) continuam valendo.

\subsection{Dinâmica do corpo rígido}

Um \textit{corpo rígido} é um corpo não pontual (tem dimensões) que
não se deforma, o qual é uma idealização. O fato de um corpo rígido
não se deformar equivale a dizer que a distância entre dois pontos
quaisquer do corpo é uma constante.

Para estudar a dinâmica de um corpo rígido, podemos dividir ele em $n$
partes pequenas e assim considerar o corpo rígido como um sistema de
$n$ corpos. Se a massa da $i$-ésima parte é $\Delta m_i$, sua posição
$\mathbf{r}_i$ e sua velocidade $\mathbf{v}_i$, o momento linear total
do sistema é
$$\mathbf{p}=\sum_{i=1}^n\dot{\mathbf{r}}_i\Delta m_i$$
e o momento angular total do sistema é
$$\mathbf{L}=\sum_{i=1}^n(\mathbf{r}_i\times\dot{\mathbf{v}}_i)\Delta m_i\,.$$
Fazendo o número de partes $n$ tender ao infinito e ao mesmo tempo a
massa das partes tender a zero, as somas anteriores tornam-se
integrais. Dessa maneira, o momento linear do corpo rígido é dado por
$$\mathbf{p}=\int \mathbf{v}\,dm$$
e seu momento angular por
$$\mathbf{L}=\int \mathbf{r}\times\mathbf{v}\,dm\,,$$
onde as integrais são sobre região limitada pelas dimensões do corpo
rígido. Com essas definições, as equações~(\ref{eq:6}) e~(\ref{eq:7})
valem para o corpo rígido.
\end{document}