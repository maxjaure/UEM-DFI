\documentclass[11pt,a4paper]{article}
\usepackage{microtype}
\usepackage{amsmath}
\usepackage{amsfonts}
\usepackage{amsthm}

\newcommand{\dpar}[1]{\left(#1\right)}

\newcommand{\N}{\mathbb{N}}
\newcommand{\R}{\mathbb{R}}

\theoremstyle{definition}
\newtheorem{ex}{Exercise}

\title{Physics 3: Class 6}
\author{Max Jauregui}

\begin{document}
\maketitle

\section{Electric potential}

Let us consider a system of charges and let us suppose that we want to
bring a particle with charge $q$ from an infinite distance to a
position $\mathbf{r}$ following a trajectory $C$. The necessary energy
to perform this task, denoted by $U(q,\mathbf{r})$, is proportional to
$q$. Then, we can define an energy per unit of charge, which is called
the \emph{electric potential} of the system, by
$$V(\mathbf{r})=\frac{U(q,\mathbf{r})}{q}\,.$$
The unit of electric potential in the SI is the volt (V).

In terms of the electric force, the energy $U(q,\mathbf{r})$ can be
written as
$$U(q,\mathbf{r})=-\int_C\mathbf{F}_e\cdot d\mathbf{r}\,.$$
Since $\mathbf{F}_e=q\mathbf{E}$, where $\mathbf{E}$ is the electric
field generated by the system, we have
$$U(q,\mathbf{r})=-q\int_C \mathbf{E}\cdot d\mathbf{r}\,.$$
This implies that the electric potential of the system of charges is
given by
\begin{equation}
  \label{eq:1}
  V(\mathbf{r})=-\int_C\mathbf{E}\cdot d\mathbf{r}\,.
\end{equation}

The electric potential of a particle with charge $Q$, localized at the
origin of our coordinate system, at a point $\mathbf{r}$ is
\begin{eqnarray*}
  V(\mathbf{r})&=&-\int_C\frac{Q}{4\pi \epsilon_0}\frac{\mathbf{r'}}{(r')^3}\,\cdot d\mathbf{r}'\\
               &=&-\int_{t_1}^{t_2}\frac{Q}{4\pi \epsilon_0[r'(t)]^3}\mathbf{r'(t)}\cdot\frac{d\mathbf{r}'}{dt}\,dt\\
               &=&-\int_{t_1}^{t_2}\frac{Q}{4\pi \epsilon_0[r'(t)]^3}r'(t)\frac{dr'}{dt}\,dt\\
               &=&-\int_{\infty}^{r}\frac{Q}{4\pi \epsilon_0(r')^2}\,dr'\,.
\end{eqnarray*}
Therefore,
$$V(\mathbf{r})=\frac{Q}{4\pi\epsilon_0}\frac{1}{r}\,.$$

If the particle of charge $Q$ is at a position $\mathbf{r'}$, the
electric potential at a point $\mathbf{r}$ will be given by
$$V(\mathbf{r})=\frac{Q}{4\pi\epsilon_0}\frac{1}{|\mathbf{r}-\mathbf{r}'|}\,.$$

If we have a system of $n$ charges $q_1,\ldots,q_n$, the electric
potential at a point $\mathbf{r}$ will be given by
$$V(\mathbf{r})=\sum_{i=1}^n\frac{q_i}{4\pi\epsilon_0}\frac{1}{|\mathbf{r}-\mathbf{r}_i|}\,.$$
In the case of a continuous distribution of charges with charge
density $\rho(\mathbf{r})$, the electric potential at a point
$\mathbf{r}$ will be
$$V(\mathbf{r})=\int_{\R^3}\frac{\rho(\mathbf{r}')}{4\pi\epsilon_0|\mathbf{r}-\mathbf{r}'|}\,dV'\,.$$

Let us consider a charged particle that is initially at a position
$\mathbf{r}_1$ and moves slowly to a position $\mathbf{r}_2$
describing a trajectory $C$ under the action of an external electric
field $\mathbf{E}$. Then, it follows from Eq.~(\ref{eq:1}) that
$$\int_C\mathbf{E}\cdot d\mathbf{r}=V(\mathbf{r}_1)-V(\mathbf{r}_2)\,.$$
In particular, if $\mathbf{r}_1=\mathbf{r}_2$, we have
$$\oint_C\mathbf{E}\cdot d\mathbf{r}=0\,.$$

\end{document}