\documentclass[12pt,a4paper]{article}
\usepackage{microtype}
\usepackage{amsfonts}
\usepackage{amsthm}

\newcommand{\N}{\mathbb{N}}
\newcommand{\R}{\mathbb{R}}

\theoremstyle{definition}
\newtheorem{ex}{Exercise}[section]

\title{Notes of elementary physics 3}
\author{Max Jauregui}

\begin{document}
\maketitle

\section{Electric force}

\subsection{Electric charge}

After rubbing two identical bars of glass with a two identical pieces
of silk, we observe that, when we try to put the two bars together, a
force appears and tries to separate the bars. In a similar way, when
we try to put the two pieces of silk together, another force appears
and tries to separate the two pieces. However, if we put the bar and
the piece of silk at a close distance, a force appears and tries to
join the two objects. From this experiment, we can conclude that there
exists a property of matter, like mass, called \emph{electric
  charge}. In fact, the experiment reveals that there are two types of
electric charge, which are called \emph{positive} and
\emph{negative}. Two charges with equal signs repel each other and two
charges with opposite signs attract each other.

\end{document}